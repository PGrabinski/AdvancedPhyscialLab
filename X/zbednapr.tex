\subsubsection{Metody wyznaczania - Efekt fotoelektryczny}
Ponieważ stała Plancka jest jedną z podstawowych stałych w przyrodzie, istnieje wiele metod jej wyznaczania:
	W 1914 Einstein opublikował teoretyczny opis efektu fotoelektrycznego. Był to kolejny krok w kierunku do powstania mechaniki kwantowej. W zjawisku tym dochodzi do wybijania elektronów z powierzchni metalu przez fotony. Możemy skonstruować układ pomiarowy w ten sposób, by móc za pomocą napięcia oporowego mierzyć energię wybitych elektronów. Gdy znajdziemy wartości napięcia granicznego - takiego przy którym przestajemy rejestrować przepływ prądu. Możemy wyznaczyć energię wybijanych elektronów. Dalej zapisując równanie energetyczne tego procesu:
	\begin{align*}
	h\nu=eU_{gr}+W\qquad\Rightarrow\qquad h=\frac{eU_{gr}+W}{\nu}
	\end{align*}
	Gdzie $\nu$ zależy od promieniowania jakim oświetlamy układ, a $W$ to praca wyjścia czyli stała materiałowa.